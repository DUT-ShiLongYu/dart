\section{LCP Formulation}
<<<<<<< HEAD
=======
The problem of contact handling is based on three types of
constraints: normal direction constraints, directional friction constraints,
and friction cone constraints. Together with equations of
motion described earlier, these three sets of constraints constitute a
LCP for unknown variables $\dot{\vc{q}}$, $\vc{f}_n$, $\vc{f}_d$, and
auxiliary variables $\lambda$.

\subsection{Normal Direction Constraints}
\begin{packed_item}
\item{expression 1:} $\vc{f}_n \geq 0$
\item{expression 2:} $N \dot{\vc{q}} \geq 0$
\item{expression 3:} $(N \dot{\vc{q}})^T\vc{f}_n = 0$
\end{packed_item}
Expression 1 ensures there is no pulling force. Expression 2 prevents
penetration by enforcing a nonnegative normal velocity at
contact. Expression 3 constrains the normal force based on the
velocity. If $N \dot{\vc{q}} > 0$, then $\vc{f}_n = 0$ (in
takeoff). If $\vc{f}_n > 0$ then $N \dot{\vc{q}} = 0$ (in contact).

\subsection{Directional Friction Constraints}
\begin{packed_item}
\item{expression 4:} $\vc{f}_d \geq 0$
\item{expression 5:} $B^T\dot{\vc{q}} + E\lambda \geq 0$
\item{expression 6:} $(B^T\dot{\vc{q}} + E\lambda)^T\vc{f}_d = 0$
\end{packed_item}
Here $E \in R^{pd \times p}$ is a binary matrix whose structure is
defined as follows: 
\begin{equation}
\label{eq:ematrix}
E = \left[\begin{matrix}\vc{e}_1 & & \\  & \ddots & \\  & & \vc{e}_p \end{matrix}\right]
\end{equation}
where $\vc{e}$ is a vector of ones in
$R^d$. Additionally, $\lambda$ is a vector in $R^p$ that contains all
of the auxiliary variables.
\begin{equation}
\label{eq:lambda}
\lambda = \left[\begin{matrix}\lambda_1 \\  \vdots \\ \lambda_p\end{matrix}\right]
\end{equation}

The goal of directional friction constraints is to ensure that contact
slipping on the surface is in the opposite direction of the friction
force. Let us first exam the first term of Expression 5,
$B^T\dot{\vc{q}}$, the velocity at contact projected onto each friction
cone basis vector. The projection will end up in one of the three cases:
\begin{enumerate}
\item The projected vector is closer to one of the bases than others. The most
  negative element in $B^T\dot{\vc{q}}$ is unique.
\item The projected vector is right in the middle of two basis
  vectors. The two most negative elements in $B^T\dot{\vc{q}}$ are the
  same.
\item The projected vector is a zero vector.
\end{enumerate}
Because the smallest possible value for any element
in $B^T\dot{\vc{q}} + E\lambda$ is zero (by Expression 5), the first case has at most
one zero element in $B^T\dot{\vc{q}} + E\lambda$. Assuming the index
of that zero element is $i$, Expression 5 and Expression 6 together
state that $\vc{f}_d$ must be a zero vector except for the $i$-th
element. This nonzero element in $\vc{f}_d$ determines the direction
of the friction force while the corresponding $i$-th element in
$B^T\dot{\vc{q}} + E\lambda$ indicates the most negative projection of
tangent velocity. Therefore, given a set of basis directions, the
friction force is indeed in the most opposite direction of contact
slipping. The second case rarely happens, but when it does we can
arbitrary pick one of the two bases to break the tie and apply the
same reasoning as the first case. The third case indicates either
static contact (when $\lambda = 0$) or contact breakage (when $\lambda
> 0$). It will be more clear after we introduce friction cone
constraints.

Is it valid to choose a large positive $\lambda$ such that all the
elements of $B^T\dot{\vc{q}} + E\lambda$ are greater than zero? It
does not seem to violate any of the expressions here, but we will see
the implication of that in the next subsection.

\ignorethis{
Assuming
there exists one zero element in $B^T\dot{\vc{q}} + E\lambda$ whose
index is $i$, the Expression 5 and Expression 6 together state that
$\vc{f}_d$ must be a zero vector except for the $i$-th element. This
nonzero element in $\vc{f}_d$ determines the direction of the friction
force. Since $B^T\dot{\vc{q}}$ is the velocity at contact projected
onto friction cone bases, the $i$-th element of $B^T\dot{\vc{q}} +
E\lambda$ indicates the most negative projection of tangent
velocity. Therefore, given a set of basis directions, the friction
force is indeed in the most opposite direction of contact slipping. Is
it valid to choose a large positive $\lambda$ such that all the
elements of $B^T\dot{\vc{q}} + E\lambda$ are greater than zero? The
answer is no but to see that we need to also consider friction cone
constraints.}

\subsection{Friction Cone Constraints}
\begin{packed_item}
\item{expression 7:} $\lambda \geq 0$
\item{expression 8:} $\mu\vc{f}_n - E^T\vc{f}_d \geq 0$
\item{expression 9:} $\lambda^T(\mu\vc{f}_n - E^T\vc{f}_d) = 0$
\end{packed_item}
Friction cone constraints describe the switch condition between static
state and slipping state of a contact. From the previous section, we
know that at most one element of $\vc{f}_d$ can be nonzero for the first (and
the second) case. Therefore, Expression 8 states that the ratio of tangent
contact force to normal contact force must be less than or equal to
the friction coefficient $\mu$. If the contact force is within the
friction cone (inequality case in Expression 8), $\lambda$ must be
zero by Expression 9. When $\lambda$ is zero, the corresponding
Expression 5 becomes $B^T\dot{\vc{q}} \geq 0$. Because the bases of
friction cones are arranged in pairs of opposite directions (\eg
$\vc{d}_1 = -\vc{d}_5$, $\vc{d}_2 = -\vc{d}_6$), the only way for all
elements to be nonnegative is when $B^T\dot{\vc{q}}$ is a zero vector
(no slipping), hence the friction cone condition.

Now, we can go back to the question about validity of selecting a
large positive $\lambda$ such that all the elements of
$B^T\dot{\vc{q}} + E\lambda$ are greater than zero. If we do so,
$\vc{f}_d$ will be all zeros by Expression 6, which leads to
$\mu\vc{f}_n \geq 0$ by Expression 8. If $\vc{f}_n > 0$, $\lambda$
must be zero by Expression 9, contradicting the assumption that
$\lambda$ is a large positive value. Therefore, $B^T\dot{\vc{q}} +
E\lambda$ can be greater than zero only when the contact is broken
($\vc{f}_n = 0$). As long as the contact exists, $\lambda$ will always
be either zero or a positive value that makes the most negative
element of $B^T\dot{\vc{q}} + E\lambda$ exactly zero.

\ignorethis{
 Recall that at most one element
of $\vc{f}_d$ can be nonzero when $B^T\dot{\vc{q}} + E\lambda$ is not
a zero vector \footnote{In a special case when the tangent velocity is
  right in the middle of two basis directions, there are two nonzero
  elements in $\vc{f}_d$. We can arbitrarily pick one of the two
  bases.}. Therefore, Expression 8 states that the ratio of tangent
contact force to normal contact force must be less than or equal to
the friction coefficient $\mu$. If the contact force is within the
friction cone (inequality case in Expression 8), $\lambda$ must be
zero by Expression 9. When lambda is zero, the corresponding
Expression 5 becomes $B^T\dot{\vc{q}} \geq 0$. Because the bases of
friction cones are arranged in pairs of opposite directions (\eg
$\vc{d}_1 = -\vc{d}_5$, $\vc{d}_2 = -\vc{d}_6$), the only way for all
elements to be nonnegative is when $B^T\dot{\vc{q}}$ is a zero vector
(no slipping), hence the friction cone condition.

Now, we can go back to the question about validity of selecting a large positive
$\lambda$ such that all the elements of $B^T\dot{\vc{q}} + E\lambda$
are greater than zero. If we do so, $\vc{f}_d$ will be all zeros by
Expression 6. Consequently,
Expression 8 becomes $\mu\vc{f}_n -
E^T\vc{f}_d > 0$, which leads to $\lambda = 0$ by Expression 9. This
contradicts our assumption that $\lambda$ is a large positive
value. Therefore, if we consider all the constraints together,
$\lambda$ will always be either zero or a positive value that makes
the most negative element of $B^T\dot{\vc{q}} + E\lambda$ exactly
zero.}

\subsection{LCP}
Putting all the constraints together, we can construct the following linear system of equations:
\begin{equation}
\label{eq:bigsystem}
\begin{array}{cc}
\left[\begin{matrix}0 \\\vc{a} \\ \vc{b} \\ \vc{c} \end{matrix}\right] =
\left[\begin{matrix}M & -\Delta t N & -\Delta t B & 0 \\ N^T & 0 & 0 & 0 \\ B^T & 0 & 0 & E \\ 0 & \mu & -E^T & 0\end{matrix}\right]\left[\begin{matrix}\vc{\dot{\vc{q}}} \\ \vc{f}_n \\ \vc{f}_d \\ \lambda\end{matrix}\right] + \left[\begin{matrix}-\tau^* \\ 0 \\ 0 \\0\end{matrix}\right] \\
\begin{matrix}
\left[\begin{matrix}\vc{f}_n \\ \vc{f}_d \\ \lambda\end{matrix}\right] \geq 0, & 
\left[\begin{matrix}\vc{a} \\ \vc{b} \\ \vc{c}\end{matrix}\right] \geq 0, &
\left[\begin{matrix}\vc{a} \\ \vc{b} \\ \vc{c}\end{matrix}\right]^T\left[\begin{matrix}\vc{f}_n \\ \vc{f}_d \\ \lambda\end{matrix}\right] = 0
\end{matrix}
\end{array}
\end{equation}
The first row of the system is based on Equation
\ref{eq:changemotionequations5}. The remaining three rows, as well as
the constraints, encapsulate the nine LCP conditions described
above. Unfortunately, the construction described is in MLCP (Mixed
LCP) form. To convert it to standard form, we need to make a few
adjustments.

\ignorethis{
\section{LCP Formulation}
>>>>>>> karen
Recall that we need $V_n$ (the relative velocity), which is given by the following expression:
\begin{equation}
\label{eq:vn0}
V_n = N^T\dot{q}
\end{equation}
Expanding this out using the example problem yields the following:
\begin{equation}
\label{eq:vn1}
V_n = \left[\begin{matrix}\vec{N_{21}}^TJ_{21} & \vec{N_{12}}^TJ_{12} & 0 \\ 0 & \vec{N_{32}}^TJ_{32} & \vec{N_{23}}^TJ_{23}\end{matrix}\right]\left[\begin{matrix}\dot{q_1} \\ \dot{q_2} \\ \dot{q_3}\end{matrix}\right]
\end{equation}
\begin{equation}
\label{eq:vn2}
V_n = \left[\begin{matrix}\vec{N_{21}}^TJ_{21}\dot{q_1} + \vec{N_{12}}^TJ_{12}\dot{q_2} \\ \vec{N_{32}}^TJ_{32}\dot{q_2} + \vec{N_{23}}^TJ_{23}\dot{q_3}\end{matrix}\right]
\end{equation}
Since $\vec{N_{ij}} = -\vec{N_{ji}}$, we can reduce this further:
\begin{equation}
\label{eq:vn3}
V_n = \left[\begin{matrix}\vec{N_{21}}^T(J_{21}\dot{q_1} - J_{12}\dot{q_2}) \\ \vec{N_{32}}^T(J_{32}\dot{q_2} - J_{23}\dot{q_3})\end{matrix}\right]
\end{equation}

\subsection{Normal Direction Constraints}
We now begin revisiting the constraints for LCP.
\begin{packed_item}
\item $\vec{f_n} \geq 0$
\item $\vec{v_n} \geq 0$
\item $\vec{v_n}^T\vec{f_n} = 0$
\end{packed_item}
The first expression ensures there is no pulling force. The second expression prevents penetration with the contact surface. The third expression constrains the normal force based on the velocity. If $\vec{v_n} > 0$, then $\vec{f_n} = 0$ (in takeoff). If $\vec{v_n} = 0$, then $\vec{f_n} \geq 0$ (in contact).

\subsection{Friction Cone Constraints}
\begin{packed_item}
\item $\vec{\lambda} \geq 0$
\item $\mu\vec{f_n} - E^T\vec{f_d} \geq 0$
\item $\lambda(\mu\vec{f_n} - E^T\vec{f_d}) = 0$
% Should this be \vec{lamba}?
\end{packed_item}
Here $E$ is a $(cd \times c)$ matrix where $d$ is the number of contact (friction cone) directions and $c$ is the number of contacts (as defined above). As an example, the structure of $E$ for a problem with $(d = 3)$ contact directions and $(c = 2)$ contacts would be as follows:
\begin{equation}
\label{eq:ematrix}
E = \left[\begin{matrix}1 & 0 \\ 1 & 0 \\ 1 & 0 \\ 0 & 1 \\ 0 & 1 \\ 0 & 1\end{matrix}\right]
\end{equation}
Additionally, $\vec{\lambda}$ is a vector of length $c$ that contains all of the $\lambda$ values.
\begin{equation}
\label{eq:lambda}
\vec{\lambda} = \left[\begin{matrix}\lambda_1 \\ \lambda_2 \\ .. \\ \lambda_c\end{matrix}\right]
\end{equation}

\subsection{Directional Friction Constraints}
\begin{packed_item}
\item $\vec{f_d} \geq 0$
\item $B^T\dot{q} + E\vec{\lambda} \geq 0$
\item $(B^T\dot{q} + E\vec{\lambda})^T\vec{f_d} = 0$
\end{packed_item}
From this, we can construct the following linear system of equations:
\begin{equation}
\label{eq:bigsystem}
\begin{array}{cc}
\left[\begin{matrix}0 \\ a \\ b \\ c \end{matrix}\right] = \left[\begin{matrix}M & -N & -B & 0 \\ N^T & 0 & 0 & 0 \\ B^T & 0 & 0 & E \\ 0 & \mu & -E^T & 0\end{matrix}\right]\left[\begin{matrix}\vec{\dot{q}} \\ \vec{f_n} \\ \vec{f_d} \\ \vec{\lambda}\end{matrix}\right] + \left[\begin{matrix}-\tau^* \\ 0 \\ 0 \\0\end{matrix}\right] \\
\begin{matrix}
\left[\begin{matrix}\vec{f_n} \\ \vec{f_d} \\ \vec{\lambda}\end{matrix}\right] \geq 0, & 
\left[\begin{matrix}a \\ b \\ c\end{matrix}\right] \geq 0, &
\left[\begin{matrix}a \\ b \\ c\end{matrix}\right]^T\left[\begin{matrix}\vec{f_n} \\ \vec{f_d} \\ \vec{\lambda}\end{matrix}\right] = 0
\end{matrix}
\end{array}
\end{equation}
The first row of the system is based on Equation \ref{eq:changemotionequations5}. The remaining three rows, as well as the constraints, encapsulate the nine LCP conditions described above.
\\
\\
Here $(a, b, c)$ are the results of the following constraints: 
\begin{packed_item}
\item $\vec{v_n} \geq 0$
\item $\mu\vec{f_n} - E^T\vec{f_d} \geq 0$
\item $B^T\dot{q} + E\vec{\lambda} \geq 0$
\end{packed_item}
The expression for $\vec{v_n}$ comes from solving for the relative velocity in Equation \ref{eq:vn3}.
\\
\\
<<<<<<< HEAD
Unfortunately, the construction described is in MLCP (Mixed LCP) form. To convert it to standard form, we need to make a few adjustments.
=======
Unfortunately, the construction described is in MLCP (Mixed LCP)
form. To convert it to standard form, we need to make a few
adjustments.
}
>>>>>>> karen
